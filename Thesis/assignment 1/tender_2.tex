%%%%%%%%%%%%%%%%%%%%%%%%%%%%%%%%%%%%
%% PREAMBLE 
%%%%%%%%%%%%%%%%%%%%%%%%%%%%%%%%%%%%

\documentclass[11pt]{article}

% nice clickable URLs
\usepackage{url} 

% page margins 
\setlength{\paperwidth}{494pt} % A4
\setlength{\textwidth}{485pt}
\setlength{\hoffset}{0pt}
\setlength{\oddsidemargin}{-5pt}
\setlength{\paperheight}{846pt} % A4
\setlength{\textheight}{700pt}
\setlength{\voffset}{0pt}
\setlength{\topmargin}{-50pt}
\setlength{\headheight}{0pt}
\setlength{\headsep}{25pt}
\setlength{\parindent}{0pt} % {20pt}
\setlength{\parskip}{4pt plus 2pt minus 1pt}


\begin{document}

%%%%%%%%%%%%%%%%%%%%%%%%%%%%%%%%%%%%
%% HEADING 
%%%%%%%%%%%%%%%%%%%%%%%%%%%%%%%%%%%%
\rule{\textwidth}{1pt}

\textbf{Causal discovery in the context of protein signalling networks} \hfill 2017
\rule{\textwidth}{1pt}
\vspace*{20pt}

%%%%%%%%%%%%%%%%%%%%%%%%%%%%%%%%%%%%
%% ENTER DETAILS
%%%%%%%%%%%%%%%%%%%%%%%%%%%%%%%%%%%%

% write the homework number in place of "NUM"
\textbf{Tender for Causal discovery from multiple observational and experimental datasets in the context of protein signalling networks thesis project}

% write your name(s) in place of "NAME(S)"
\textbf{Name:} Alex Khawalid, 10634207\\
\textbf{Supervisor:} Sara Magliacane\\
\today

%  why me
Throughout my entire bachelor, I have always taken extra courses. The reason for this is that I wanted to explore and understand, at least a small part, of different fields. Since some of the extra courses I took were outside of my field of expertise, I have experience with figuring out things in a field I am not necessarily comfortable with.  My main interest is still Artificial Intelligence, but I know it can be applied to almost anything in a meaningful way.

% motivation
Last semester I finished a course called Computational Cognitive Neuroscience, this course involved simulating and understanding certain processes in the brain. The biological side of it was interesting to me, I expect that this research project's subject will be interesting to me in the same way. I'm curious about how Joint Causal Inference improves upon traditional methods.

% approach
To solve this problem I'll start out doing some literature research to properly understand the Joint Causal Inference (JCI) and causal influences in cellular signalling networks. Once I have gained a proper understanding of the task at hand I will implement JCI system, preferably in python. Python seems like a good choice since it has a lot of useful libraries. These libraries can be used for data processing or data visualisation. Furthermore Python is a language in which it is easy to make neat modular code, this makes it possible to implement the system in such a way that it is easy to adjust. When the implementation is finished, I will try to demonstrate that it can discover causal relations with smaller set of data than traditional methods. This will be done by comparing its results with other research. 

% result
The expected result of the research will be that the JCI implementation will be able to infer causal relations from a smaller set of data than traditional methods for the reconstruction of protein signalling networks.



\end{document}